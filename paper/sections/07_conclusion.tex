% -----------------------------------------------------------------------------
\section{Conclusion}
\label{sec:conclusion}
% -----------------------------------------------------------------------------

In this work, we proposed the `LCTLayer`, an adaptive layer based on the Linear Canonical Transform, as a novel component for neural network architectures, particularly Transformers. By allowing the model to learn the LCT parameters \((a,b,c)\), we provide a mechanism for discovering data-dependent transformations that generalize fixed operations like the Fourier Transform.

Our preliminary experiments with integrating `LCTLayer` into a NanoGPT-style model suggest [TODO: Briefly restate key positive finding, e.g., "promising improvements in computational throughput while maintaining competitive language modeling performance," or "a favorable speed-accuracy trade-off compared to the baseline"].

The main contributions of this paper are:
\begin{itemize}
  \item The design and PyTorch implementation of a JIT-compatible, GPU-accelerated `LCTLayer` with learnable parameters and an analytical inverse.
  \item The integration of this layer into a standard Transformer (NanoGPT) architecture.
  \item An initial empirical evaluation demonstrating [TODO: e.g., "the potential for efficiency gains" or "the adaptability of the learned transforms"].
\end{itemize}

Future work will focus on more extensive benchmarking across larger datasets and model sizes, exploring the application of 2D LCTs for vision tasks, and a deeper theoretical analysis of the learned LCT parameter spaces and their inductive biases. We also plan to investigate optimal strategies for LCT layer placement and initialization within various neural architectures.

We believe that adaptive transform layers like the LCTLayer offer a promising direction for building more efficient, expressive, and versatile deep learning models.

The \textsc{LCT} layer generalises Fourier projections and delivers consistent speed/accuracy improvements across modalities.  We release code and pre‐trained weights to facilitate adoption.