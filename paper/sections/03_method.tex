% -----------------------------------------------------------------------------
\section{The \\textsc{LCT} Layer}
\label{sec:method}
% -----------------------------------------------------------------------------
We parameterise the discrete LCT by the symplectic matrix
\[ S \;=\;
  \begin{bmatrix} a & b \\ c & d
\end{bmatrix}, \qquad ad - bc = 1. \]
Only \((a,b,c)\) are learnable; we compute $d$ on the fly via the symplectic constraint.  When $a=0$ and $c=0$ the layer reduces to an orthonormal DFT.

\subsection{Forward Pass}
The transform factorises into a chirp phase, FFT, and a second chirp:
\[ X = C_2 \, F \left( C_1 \odot x \right). \]
This admits an $\mathcal{O}(N \log N)$ implementation building on
\verb|torch.fft.fft| for GPU acceleration.

\subsection{Inverse Pass}
Because the discretisation is unitary, the backward pass is the Hermitian transpose of the forward kernel, yielding a closed‐form gradient.